\documentclass{amsart}
\usepackage{amsmath} 
\usepackage{amssymb}
\usepackage{graphicx}
\usepackage{pictex}
\author{Emanuel Berggren}
\title{Graph coloring using modular decomposition}
\newtheorem{definition}{Definition}

\begin{document}
\maketitle

\section{Abstract}

\section{Introduction}

In this paper, we investigate a new heuristic using the modular decomposition of
a graph, and if it can be combined with other heuristics. 

The modular decomposition of a graph describes the structure of the graph, by
recursively splitting it into distinct modules. These modules contain some
vertexes from the graph, and can be further split into modules that partition
the top module. Every parent vertex also has a type, that determine how its submodules
combine to form their parent module.

Modular decomposition allows for coloring of graphs in linear time with optimal
chromatic number if all of the modules are vertexes are either  series or
parallel (REFERENS). However, if any of the vertexes are prime then optimal graph coloring 
is again NP-Hard, as prime vertexes cannot be recursively colored by this
method. The question examined here, is however if one can still utilize the
modular decomposition, with graph coloring heuristics for the prime colorings.
The modular decomposition might still color some parts of the graph optimally,
and the structure it provides might provide a hint for how to apply the
heuristics.

In section (PART) the basic terms are described.

In section (SECTION) the heuristics are described

In section (SECTION) the combination strategies are described, that is how are
these heuristics applied, and how is the structure of the modular decomposition
utilised.

In section (SECTION) the data and benchmarking methods are described. The data
used is both from DIMACS benchmarks (REFERENS), and custom generated data.

Finally, the results are presented in section (SECTION)

\section{Definitions}

\subsection{Modular decomposition}

\begin{definition}[Graph module]
    Let $G = (V,E)$ be an arbitrary graph. A non-empty vertex set $X \subset V$
    is a \textit{module} of $G$ if, for every $y \in V \setminus X$,  either
    $N(y) \cap X = \emptyset$ or $X \subset N(y)$ is true. A module $M$ is
    \textit{strong} if it does not overlap with any other module $M'$, i.e, if 
    $M \cap M' \in \{M,M',\emptyset \}$.
\end{definition}
   
%modular decomposition

\begin{definition}[Modular decomposition]
    The modular decomposition of a graph $G = (V,E)$ is a directed, rooted, vertex labeled tree
    $(\widetilde{T},t)$, where each of its vertices is associated with a strong
    module $X$ of $G$, and a label $t$ distinguishing  three cases:

    Series: The induced subgraph $G[X]$ constructed by graph union of the
    induced subgraphs of it's children.
    Parallel: The induced subgraph $G[X]$ is constructed by graph (SOMETHING) of
    the induced subgraph of it's children
    Prime: The induced subgraph $G[X]$ is constructed by some other means from
    it's children
    %
    %
    %
    And where the $V = X'$ when $X'$ is the strong component associated with the
    root node of the modular decomposition.
\end{definition}

The modular decomposition of a graph is unique (REFERENS), which means that the
construction of the modular decomposition doesn't vary and need not be a part of
the heuristic.

\begin{definition}[Quotient graph]

\end{definition}

\begin{definition}[Partially color graphed]


\end{definition}

\section{Heuristics}

%ska man ta med sånt alla vet
A graph coloring heuristic is an algorithm for coloring a graph, that doesn't
necessarily give an optimal coloring, but uses various methods to approximate a
good coloring. There are numerous graph coloring algorithms (REFERNENS) and is
one of the most well studies parts of graph theory (REFERENS). 

However, in order of be useful for the different STRATEGIES, a slightly
different problem is solved by these algorithms, and that is to colorize a
partially colored graph.

The algorithms used are therefore slightly modified versions of the 
Greedy (REFERENS), dsatur (REFERENS) and Recursive largest first (REFERENS). For
completeness, the algorithms are described fully here, with their modifications.

\section{General coloring algorithm}



\section{Strategies}


In the case where the whole graph doesn't contain any prime modules, coloring
can is easy (REFERENS), it is when coloring prime modules where we have to apply
the heuristics, and these different methods are described in this part.


\subsection{Whole prime coloring}

With this strategy, the prime vertex with the strong module $X$ of $G$, the
whole induced subgraph $G[X]$ is colored with the heuristic.

This method is the simplest combining strategy, but worth noting is that is
equivalent to coloring the whole graph using the heuristic in the case where the
root node is prime, and therefore only offers a possible improvement to existing
heuristics when the graph isn't prime.

\subsection{Quotient recoloring}

With this method, all of the children for the prime vertex are first colored,
and then the quotient graph for the child modules are constructed. This quotient
graph is then colored. Then all of the child modules that have the same color
are recolored to use the same colors among the largest set of colors for the
modules.

%formal description


\subsection{Largest child first}

The largest child module is first colored with the same coloring algorithm, and
then the heuristic is applied on the partially colored induced subgraph 
$G[X]$.

%formal description

\subsection{Data}

The test sets used are part graphs from the DIMACS benchmark set (REFERENS), 
and part graphs generated that aren't prime.

All of the graphs from the DIMACS benchmark sets are difficult to color graphs,
and also have modular decomposition where the root vertex is prime. However, the
modular decomposition might provide a more efficient way to color graphs where
only some of the child nodes are prime. 

These graphs are generated by (GENERATION METHOD)

\subsection{Results and evaluation}

lorem ipsum

\includegraphics[width=\textwidth]{Results.png}


\end{document}
