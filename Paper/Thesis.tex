
\documentclass{amsart}
\usepackage{amsmath} 
\usepackage{amssymb}
\usepackage{graphicx}
\author{Emanuel Berggren}
\title{Graph coloring with modular decomposition}
\newtheorem{definition}{Definition}

\begin{document}
\maketitle

\section{Abstract}

\section{Definitions}



\section{Modular decomposition}

\begin{definition}
    Let $G = (V,E)$ be an arbitrary graph. A non-empty vertex set $X \subset V$
    is a \textit{module} of $G$ if, for every $y \in V \setminus X$,  either
    $N(y) \cap X = \emptyset$ or $X \subset N(y)$ is true. A module $M$ is
    \textit{strong} if it does not overlap with any other module $M'$, i.e, if 
    $M \cap M' \in \{M,M',\emptyset \}$.
\end{definition}
   
%modular decomposition

\begin{definition}
    The modular decomposition of a graph $G = (V,E)$ is a directed, rooted, vertex labeled tree
    $(\widetilde{T},t)$, where each of its vertices is associated with a strong
    module $X$ of $G$, and a label $t$ disinguishing  three cases:
    %
    %
    %
    And where the $V = X'$ when $X'$ is the strong component associated with the
    root node of the modular decomposition.
\end{definition}

\end{document}
