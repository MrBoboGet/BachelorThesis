\documentclass{amsart}
%\usepackage{amsmath} 
%\usepackage{amssymb}
%\usepackage{graphicx}
%\usepackage{pictex}
%\usepackage[noend]{algorithmic}
%\usepackage{algorithm}
%\renewcommand{\algorithmiccomment}[1]{\hfill$\rhd???$\textit{#1}}
%\usepackage{mathtools}
%\providecommand{\keywords}[1]{\textbf{\textit{Keywords: }} #1}
%\theoremstyle{definition}
%\newtheorem{definition}{Definition}


% COPY PASTE

\usepackage{authblk}
%\usepackage[numbers,sort&compress]{natbib}
%\usepackage[numbers, sort&compress]{natbib}



\usepackage[colorlinks]{hyperref}
\hypersetup{
	citecolor=blue,
   linkcolor=red,
}

\usepackage{amsmath, amssymb, amsfonts, amsfonts, amsthm,latexsym}
\usepackage[noend]{algorithmic}
\usepackage{algorithm}
\renewcommand{\algorithmiccomment}[1]{\hfill$\rhd???$\textit{#1}}
\usepackage{graphics}
\usepackage{enumerate}
\usepackage[usenames]{color}
\usepackage{mathtools}
\usepackage[normalem]{ulem}
\newtheorem{theorem}{Theorem}[section]
\newtheorem{proposition}[theorem]{Proposition}%[section]
\newtheorem{lemma}[theorem]{Lemma}%[section]
\newtheorem{definition}{Definition}[section]
\newtheorem{corollary}[theorem]{Corollary}%[section]
\newtheorem{remark}[theorem]{Remark}%[section]
\newtheorem{problem}{Problem}[section]
\newtheorem{observ}{Observation}
\usepackage{url}
\RequirePackage{hyperref}
\DeclareMathOperator{\sgn}{sgn}

\DeclareSymbolFont{yhlargesymbols}{OMX}{yhex}{m}{n}
\DeclareMathAccent{\overarc}{\mathord}{yhlargesymbols}{"F3}

%%%%% SOME LOW LEVEL STUFF NEEDED FOR SPECIAL SYMBOLS 
\makeatletter
\def\moverlay{\mathpalette\mov@rlay}
\def\mov@rlay#1#2{\leavevmode\vtop{%
    \baselineskip\z@skip \lineskiplimit-\maxdimen
    \ialign{\hfil$\m@th#1##$\hfil\cr#2\crcr}}}
\newcommand{\charfusion}[3][\mathord]{
  #1{\ifx#1\mathop\vphantom{#2}\fi
    \mathpalette\mov@rlay{#2\cr#3}
  }
  \ifx#1\mathop\expandafter\displaylimits\fi}
\DeclareRobustCommand\bigop[1]{%
  \mathop{\vphantom{\sum}\mathpalette\bigop@{#1}}\slimits@
}
\newcommand{\bigop@}[2]{%
  \vcenter{%
    \sbox\z@{$#1\sum$}%
    \hbox{\resizebox{\ifx#1\displaystyle.9\fi\dimexpr\ht\z@+\dp\z@}{!}{$\m@th#2$}}%
  }%
}
\makeatother
%%%%% END LOWLEVEL USER DEFINITION 
\newcommand{\bigjoin}{\bigop{\triangledown}}
\DeclareMathOperator{\join}{\triangledown}
\newcommand{\cupdot}{\charfusion[\mathbin]{\cup}{\cdot}}
\DeclareMathOperator{\bigcupdot}{\charfusion[\mathop]{\bigcup}{\cdot}}
\definecolor{jade}{rgb}{0.0, 0.66, 0.42}
\newcommand{\child}{\mathsf{child}}
\DeclareMathOperator*{\argmax}{arg\,max}
\newcommand{\Pmax}{\mathrm{Pmax}}
\newcommand{\T}{\widetilde{T}}
\renewcommand{\t}{\widetilde{t}}

%\definecolor{darkorchid}{rgb}{0.6, 0.2, 0.8}

%\newcommand{\TODO}[1]{\begingroup\color{red}#1\endgroup}
\newcommand{\AX}[1]{\textnormal{#1}}
%%\newcommand{\NEW}[1]{\begingroup\color{blue}#1\endgroup}
%\newcommand{\dv}[1]{\begingroup\color{jade}#1\endgroup}
%\newcommand{\OLD}[1]{\begingroup\small	\color{green}#1\endgroup}
%\newcommand{\PFS}[1]{\begingroup\color{blue}#1\endgroup}
%\newcommand{\mhr}[1]{\begingroup\color{darkorchid}#1\endgroup}
%\newcommand{\mh}[1]{\begingroup\color{magenta}#1\endgroup}
%\newcommand{\mg}[1]{\begingroup\color{cyan}#1\endgroup}

%%\journal{European Journal of Combinatorics} 

\providecommand{\keywords}[1]{\textbf{\textit{Keywords: }} #1}

 % END COPY PASTE


\usepackage[backend=bibtex]{biblatex}
\usepackage{todonotes}
\addbibresource{References.bib}


\newcommand{\algorithmicbreak}{\textbf{break}}
\newcommand{\BREAK}{\STATE \algorithmicbreak}
\newcommand{\algorithmautorefname}{Algorithm}
\author{Emanuel Berggren}
\title{Graph coloring using modular decomposition}

\begin{document}
\maketitle

\section{Abstract}

\todo[noline]{Execution time is difficult to measure, but the difference in
results here are quite large, so probably something worth measuring. Might need
to update my implementation however}

Graph coloring is one of the most common and studied problems in computer
science. It has application in many different areas, such as (EXAMPLES), but
being NP-Hard means that optimal solutions are infeasible, meaning that most
practical problems require the use of heuristics. This thesis examines a
heuristic based on the modular decomposition of a graph, a way to split the grab
into distinct partitions of vertices where some parts can be colored optimally,
and some parts require other heuristics. This strategy of coloring with the
modular decomposition was found to give a small increase in performance for the
TabuCol heuristic, but no increase for the other tested heuristics.

%but the modular decomposition could have applicability in
%designing parallel coloring algorithms or to improve execution time with a
%practical linear time modular decomposition algorithm.



\section{Introduction}

In this thesis, we investigate a new heuristic using the modular decomposition of
a graph, that is used alongside other traditional graph coloring heuristics.

The modular decomposition of a graph describes the structure of the graph, by
recursively splitting it into distinct modules. A module is a set of vertices in
a graph, which share a common neighbourhood of vertices among vertices outside
of the module. Modules also form a hierarchy, which means that modules can be
further subdivided into smaller modules. This representation of a graph can be
described with a tree, where every vertex has one of three possible labels
describing how it's child modules can be combined to form the graph induced by
it's parent module. A module being series means that the parent is constructed
with a disjoint union of it's children, a module being parallel means that it
can be constructed through graph join on it's children, and prime means that the
construction is not straightforward, having an arbitrary internal structure.

Modular decomposition allows for coloring of graphs in linear time with optimal
chromatic number if all of the modules are either  series or
parallel \cite{HCL}. However, if any of the vertices are prime then optimal graph coloring 
is again NP-Hard\cite{NPHard}, as prime vertices cannot be recursively colored by this
method. The question examined here, is if one can still utilize the
modular decomposition, with graph coloring heuristics for the prime modules.
The modular decomposition might still color some parts of the graph optimally,
and the structure it provides might provide a hint for how to apply the
heuristics on the prime parts, improving performance for other heuristics.

%kanske rätt onödigt?
In \autoref{sec:Definitions} all of the required terminology and definitions is
provided, and is split into two parts, \autoref{sec:GraphBasics} has definitions that might be
familiar to most people that have worked with graphs, and
\autoref{sec:GraphModules} provide the definitions that are more specific to
this thesis.

In \autoref{sec:Heuristics} the graph coloring heuristics used are described.

In \autoref{sec:Strategies} the combination strategies are described, that is how are
these heuristics applied, and how is the structure of the modular decomposition
utilised.

In \autoref{sec:Data} the test graphs and benchmarking methods are described. The data
used is both from standard DIMACS benchmarks \cite{DIMACS}, and custom generated data.

Finally, the results are presented in \autoref{sec:Result}.

\section{Definitions}
\label{sec:Definitions}

\subsection{Graph basics}
\label{sec:GraphBasics}

The definitions used here is based of the definitions in 
\cite{GraphBasics}.

\begin{definition}[Graph]
    A graph $G = (V,E)$ is a tuple, where $V$ is the set of vertices, and $E$ is
    a set of unordered pairs, the edges, such that for all $(v,u) \in E$,
    $v \neq u$, $v \in V$ and $u \in V$.
\end{definition}
\begin{definition}[Neighbour]
    For a graph $G = (V,E)$, we say that $v \in V$ is adjacent to 
    $u \in V$ if $(v,u) \in E$. 

    The neighbourhood $N_G(v)$ for a vertex $v \in V$ in a graph $G = (V,E)$,
    is the set of vertices that are adjacent to $v$, that is 
    $N_G(v) = \{u : (u,v) \in E \}$. If $u \in N_G(v)$, we also say
    that $u$ and $v$ are neighbours.
\end{definition}
\begin{definition}[Degree]
    The degree for a vertex $v \in V$ in a graph $G = (V,E)$, denoted by 
    $deg(v)$, is the number of neighbours for $v$ in $G$, that is 
    $deg(v) = |N_G(v)|$.
\end{definition}

In many cases, one is interested in some parts of the graph, and one of the
most common ways to subset a graph is through the induced subgraph. The induced
subgraph is a graph constructed by only including a subset of the original
graphs vertices, and edges between them.

\begin{definition}[Subgraph]
    A subgraph $S = (V',E')$ of a graph $G = (V,E)$, is a graph such that
    $V' \subset V$, $E' \subset E$.
\end{definition}

\begin{definition}[Induced subgraph]
    
    For a graph $G = (V,E)$, the induced subgraph $G[X]$ for $X \subset V$, is
    the subgraph $(X,\{(u,v) : (u,v) \in E, u \in X,v \in X\})$. That
    is, a new graph only containing the vertices in $X$, and only edges between
    these vertices.

\end{definition}

Another common operation, is the graph complement. The graph complement of a
graph is graph that contains the same vertices as the original graph, but the
vertices are adjacent only if they where not adjacent in the original graph.
This also means that the union of their edge sets contains all possible edges
between the vertices.

\begin{definition}[Graph complement]
    The graph compliment $\overline{G}$ of a graph $G = (V,E)$ is the graph 
    $(V,\{ (u,v) : u \neq v, u \in V,v \in V, u \notin N_G(v) \})$.
\end{definition}

Some other common operations, is how to combine two distinct graphs. Two common,
and especially relevant for the modular decomposition further down, is the
disjoint union and graph join. The disjoint union is the most simple way to
combine graphs, and just forms a new graph that just contains the original graphs
and nothing more. 
The graph join is similar, in that it produces a new graph with all of edges
and vertices from the original graphs, but also for every vertex in a
input graph adds a new edge to the other input graphs.

\begin{definition}[Disjoint union]
    The disjoint union $\bigcup_i G_i$ for graphs $G_i = (V_i,E_i)$ where 
    $\bigcap V_i = \varnothing $ , is the graph
    $G = \left( \bigcup V_i,\bigcup E_i \right)$.
\end{definition}

\begin{definition}[Graph join]
    The graph join $\nabla G_i$ for graphs $G_i = (V_i,E_i)$ where 
    $\bigcap_i V_i = \varnothing$, is the graph $G = (\{\bigcup V_i,
    \bigcup E_i \cup \{(u,v) : u \in V_k, v \in V_j, k \neq j \})$
\end{definition}

\todo[noline]{Fix description of connected component}

One of the original applications of graph theory was to describe different ways
to walk in city (REFERENS), so the concept of a path has been central for a long
time.  A path is a sequence vertices, so that the next vertex in the sequence is
connected to the previous by an edge, and so that vertices are not repeated.
One can then imagine "walking" between vertices by moving on edges.

\begin{definition}[Path]
    A path $p$ in graph $G = (V,E)$ is a sequence of vertices $p = (v_1\cdots
    v_i)$, such that $(v_j,v_{j+1}) \in E$ for $1 \leq j \leq i-1$ and $v_j \neq v_k$ 
    for $0 \leq j,k \leq i-1$ when $j \neq k$.
\end{definition}

\begin{definition}[Connected graph]
    A graph $G = (V,E)$ is connected, if there exists a path between any two
    vertices in the graph. Otherwise, we say that a graph is disconnected.
\end{definition}

The central problem in this thesis, is the efficient coloring of a graph.
Coloring graph means that one assigns to every vertex a color, an arbitrary
value, such that no vertices that are neighbours have the same color. A classic
example of this problem is when coloring a map and wants to avoid giving
neighbouring countries the same color.  Every country can be represented by a
vertex, and edges are added if two countries share a border.

\begin{definition}[Graph coloring]
    A graph coloring $\sigma$ for a graph $G = (V,E)$ is a map from $V$ to $C$,
    where $C$ is a set of colors, such that no neighbours share the same color,
    that is $(u,v) \in E \to \sigma(u) \neq \sigma(v)$. We say that $\sigma$
    is a $k$ coloring if $|C| \leq k$.
\end{definition}
\begin{definition}[Partial coloring]
    A partial graph coloring $\sigma$ for a graph $G = (V,E)$ is a map from $V$ to $C$,
    where $C$ is a set of colors, with the special 'EMPTY' color, such that no
    neighbours share the same color, unless that coloring is empty, that is
    $(u,v) \in E \to \sigma(u) \neq \sigma(v) \vee \sigma(u) = \sigma(v) = EMPTY$. 
\end{definition}
\begin{definition}[Improper coloring]
    An improper graph coloring $\sigma$ for a graph $G = (V,E)$ is a map from $V$ to $C$,
    where $C$ is a set of colors, allowing for clashes in coloring.
\end{definition}
\begin{definition}[Chromatic number]
    The chromatic number of a graph $G = (V,E)$, denoted by $\chi(G)$,
    is the least amount of colors needed to color the graph, that is neighbours 
    can share the same color.
\end{definition}

Finding a coloring for a graph is easy, one could for example use
\autoref{alg:greedy}. Finding an optimal coloring however, a coloring using the
fewest possible colors is NP-Hard \cite{NPHard}. For this purpose, multiple
different heuristics, algorithms providing a proper coloring with as few colors
as possible but not necessarily optimal, have to be used.

\begin{definition}[Complete graph]
    A graph $K = (V,E)$ is complete if every vertex in $v \in V$ is adjacent to
    every other vertex, that is, $ (u,v) \in E \iff v \neq u$. We also denote the graph $K_n$ with the complete graph that
    has $n$ vertices.
\end{definition}

\subsection{Graph modules and cographs}
\label{sec:GraphModules}

\begin{definition}[Closure]
    A set $S$ is closed under some operator $P$, if 
    $ P(v_1,\cdots v_i) \in S$ when $v_1,\cdots v_i \in S$.
\end{definition}

\begin{definition}[Cograph]
    The set of cographs $\textbf{Co}$, is a set containing $K_1$ that is closed under 
    graph join and disjoint union. A graph $G$ is a cograph if $G \in \textbf{Co}$.
\end{definition}


\begin{definition}[Graph module]
    Let $G = (V,E)$ be an arbitrary graph. A non-empty vertex set $M \subset V$
    is a module of $G$ if, $\forall x,y \in M (N_G(x) \setminus M = N_G(y) \setminus M)  $. A module $M$ is
    strong if it does not overlap with any other module $M'$, i.e, if 
    $M \cap M' \in \{M,M',\emptyset \}$.
\end{definition}
  

Now we have sufficient terminology to describe the most central concept, the
modular decomposition. The modular decomposition is way to partition the
vertices of the graph into a tree structure, where the relation between
different modules, parts in the partitions, are known. Another way to look at
the modular decomposition, is that it's a cograph approximation of the original
graph, where parts that don't have a cograph structure are replaced with prime
modules.

\todo[noline]{Update definition}
\begin{definition}[Modular decomposition]
    The modular decomposition MD for a graph $G =(V,E)$, is the set of all
    strong modules of the graph.
\end{definition}

The modular decomposition of a graph has a number of useful properties. The
modular decomposition of a graph forms a hierarchy (REFERENS), meaning that strong modules
can be partioned into other strong modules. This hierarchy of modules has a
natural representation as a tree, the modular decomposition tree.

This tree describes the original graph in terms of a tree, where every vertex 
in the tree represents a strong module, with the root being the strong module
containing all vertices. This tree also describes how the indueced subgraph of the
module can be constructed from the indueced subgraphs of it's children.
\begin{definition}[Modular decomposition tree]
    The modular decomposition tree $(\T,\t)$ of a graph $G = (V,E)$,is a rooted
    vertex labeled tree such that every vertex is associated with a strong
    module $X$ in $G$, and with a label distinguishing 3 cases.
    \begin{enumerate}
        \item Series: $G[X]$ is disconnected.
        \item Parallel: $\overline{G[X]}$ is disconnected.
        \item Prime: $G[X]$ and $\overline{G[X]}$ is disconnected.
    \end{enumerate}
    And such that the root of the tree has $V$ as the associated vertex, and all 
    strong modules are a part of the tree.
\end{definition}

Note, a label being series in modular decomposition tree with associated module
$X$ means that the induced subgraph $G[X]$ can be constructed through graph
union on the induced subgraphs of its children's associated modules, and it being
parallel means that $G[X]$ can be constructed through graph join on the
induced subgraphs of its childrens associated modules \cite{HCL}.

As a single vertex is a module, every leaf of the modular decomposition has
a single vertex as a module. This means that a modular decomposition without
prime modules is a cograph, as it is recursively constructed by graph join and
graph union on $K_1$.

The modular decomposition of a graph is unique \cite{MDUnique}, which means that the
construction of the modular decomposition doesn't vary and need not be a part of
the heuristic.


\section{General coloring algorithm}

The algorithm used for coloring is algorithm 2 in \cite{HCL}. It provides a
optimal coloring given that the graph is a cograph. It does however not
provide a way to color the prime modules. This paper examines multiple different
ways these prime modules can be colored, split into 2 parts, a heuristic used,
and a coloring strategy. The heuristic is an ordinary coloring algorithm
providing an approximate optimal coloring, and the strategy is the way this
heuristic is applied. The different combinations of strategies and heuristic are
then compared against the baseline of applying the same heuristic on the whole
graph.

\begin{algorithm}[H]
  \caption{Modularly-minimal coloring a graph $G$ with MD tree $(T,t)$.}
  \label{alg:generic}
  \algsetup{linenodelimiter=}
  \begin{algorithmic}[1]
    \REQUIRE Graph $G$ and MD tree $(\T,\t)$
    \STATE Initialize a coloring $\sigma$ s.t.\ all $v \in V(G)$
           have different colors
      \FORALL{$u\in V^{0}(T)$ \text{in post order}}
       \IF {$u$ is parallel} 
          \STATE $\mathcal{G} \leftarrow \{G(w)\colon w\in\child(u)\}$ 
          \STATE $G^* \leftarrow \argmax_{w\in\child(v)} |\chi(G(w))|$
          \STATE $S \leftarrow \sigma(V(G^*))$ 
          \FOR {$H\in\mathcal{G}\setminus \{G^*\}$} 
             \STATE randomly choose an injective map $\phi:\sigma(H)\to S$
             \FORALL {$x\in H$}
                \STATE $\sigma(x)\leftarrow \phi(\sigma(x))$  
             \ENDFOR
          \ENDFOR
       \ELSIF{$u$ is \emph{prime}} 
          \STATE Construct a modularly-minimal coloring of $G(u)$
              with colors contained in $\sigma(G(u))$
              and adjust $\sigma$ accordingly 
       \ENDIF
    \ENDFOR
  \end{algorithmic}
\end{algorithm}

Here, the last condition is of main interest, how to modularly-minimaly 
color a prime vertex.

\section{Heuristics}
\label{sec:Heuristics}

%ska man ta med sånt alla vet
A graph coloring heuristic is an algorithm for coloring a graph, that doesn't
necessarily give an optimal coloring, but uses various methods to approximate a
good coloring.


In this section, the different heuristics used to color the graph is described,
while \autoref{sec:Strategies} describe how they are applied.
\subsection{Greedy}
The classic greedy algorithm. Greedy walks over all of the vertices in the graph
in an arbitrary order, and for every vertex assigns the first colored not shared
amongst it's neighbours. Being relatively simple, it also has fast runtime, with
a time complexity of $O(|V|+|E|)$ \cite{Constructive}

\begin{algorithm}[H]
  \caption{Greedy}
  \label{alg:greedy}
  \algsetup{linenodelimiter=}
  \begin{algorithmic}[1]
      \REQUIRE Graph $G = (V,E)$
      \STATE $V' \leftarrow \text{List containing all $v \in V$ in any order}$
      \STATE $C \leftarrow \text{List of possible colors $\{1 \cdots V| \}$ }$
      \STATE $\sigma \leftarrow \text{Initial empty coloring}$
    \FORALL{$v \in V$}
        \FORALL{$c \in C$}
            \IF {$c \neq \sigma(n) \text{for all $n \in N_G(v)$ } $}
                \STATE Update $\sigma$ so that $\sigma(v) = c$.
                \BREAK
            \ENDIF
        \ENDFOR
    \ENDFOR
  \end{algorithmic}
\end{algorithm}
\subsection{Dsatur}

Dsatur is an algorithm that is similar to Greedy, in that it walks through every
vertex and assigns it the first available color. The difference is mostly in
how this traversal is constructed. In greedy, this traversal is an arbitrary
order, but for Dsatur this traversal is constructed in a deterministic fashion,
by using the saturation degree.

\begin{definition}[Saturation degree]
    The saturation degree $sat(v)$ for a vertex $v \in V$ for a graph $G =
    (V,E)$ and a partial coloring $\sigma$ to the set of colors $C$, is the amount of unique colors among
    it's colored neighbours, that is $sat(v) = |\{c  : c = \sigma(u), u \in N_G(v),c \neq EMPTY  \}|$.
\end{definition}

The vertex are colored so that the vertex with the highest saturation degree is
colored first, alternatively maximizing the degree. Compared to Greedy, this
algorithm does multiple passes over every vertex as every new coloring changes
the saturation degree for the remaining vertices. This means that the runtime
for the algorithm is higher, being
$O((|V|+|E|)\log{|V|})$ \cite{Constructive}.

DSatur being a well studied algorithm also means that it has some variations in 
the litterature, and this presentation is based on \cite{Constructive}.

\begin{algorithm}[H]
  \caption{Dsatur}
  \algsetup{linenodelimiter=}
  \begin{algorithmic}[1]
      \REQUIRE Graph $G = (V,E)$
      \STATE $C \leftarrow \text{List of possible colors $\{1 \cdots |V| \}$ }$
      \STATE $\sigma \leftarrow \text{Initial empty partial coloring}$

      \FOR{$i$ between $0$ and $|V|$}

        \STATE $v \leftarrow$ uncolored vertex in $V$, such that $sat(v)$ is
        minimal. In case of ties, choose the $v$ that also minimizes $deg(v)$
        for the subgraph of $G$ induced by the uncolored vertices.
        \STATE $\sigma \leftarrow$ updated coloring where $\sigma(v)$ is the first
        available color in regards to $C$.
      \ENDFOR
  \end{algorithmic}
\end{algorithm}

\subsection{Recursive largest first}

Recursive largest first is more complicated than both Greedy and Dsatur, in both
execution and runtime. 

The idea behind Recursive largest first, is to create a partition of the
vertices of the graph, where the vertices in the different parts are all
non-adjacent to each other. This ensures that every partition can share the
same color. 

This construction is made by adding the vertex with the highest degree in the
induced subgraph of unpartitioned vertices to the current part that
isn't the neighbour of any vertex in the current part, and when no
such vertex exist, create a new empty part in the partition and continue. When
all of the vertices have been partitioned so is the coloring extracted by
assigning every part in the partition a unique color, and every vertex in that
part the same color.

This presentation of the algorithm is based on \cite{Constructive}.
\begin{algorithm}[H]
    \caption{Recursive largest first (RLF)}
  \algsetup{linenodelimiter=}
  \begin{algorithmic}[1]
      \REQUIRE Graph $G = (V,E)$
      \STATE $\text{Partition} \leftarrow \{\}$
      \STATE $C \leftarrow \text{List of possible colors $\{1 \cdots |V| \}$ }$
      \STATE $M \leftarrow \{\}$
      \STATE $S \leftarrow V$
      \WHILE{$|S| > 0$}
        \STATE $v \leftarrow \text{the vertex maximizing $deg(v)$ for G[S]} $
        \STATE $M \leftarrow \{v\}$
        \WHILE{TRUE}
            \STATE Candidates $\leftarrow \{\}$
            \FORALL{$v' \in G[S]$}
                \IF{$v' \neq m \wedge v' \notin N_{G[S]}(m) \text{for all $m \in
                M$}$}
                    \STATE $\text{Candidates} \leftarrow \text{Candidates}
                    \cup \{v'\}$
                \ENDIF
            \ENDFOR
            \IF{$|\text{Candidates}| = 0$}
                \BREAK
            \ENDIF
            \STATE $n \leftarrow \text{the $n \in \text{Candidates}$ maximizing $deg(n)$ in $G[S]$}$ 
            \STATE $M \leftarrow M \cup \{n\}$
            \STATE $S \leftarrow S \setminus \{n\}$
        \ENDWHILE
        \STATE Partition $\leftarrow \text{Partition} \cup \{M\}$ 
      \ENDWHILE
      \STATE Assign consecutive colors from $C$ to the partitions in Partitions,
      and then color every vertex in that partition with that color.
  \end{algorithmic}
\end{algorithm}

\subsection{TabuCol}

TabuCol is a graph algorithm that unlike the previous algorithms, doesn't
construct a coloring in a constructive fashion, but instead tries to find a
coloring for a specific number target colors.

TabuCol first forms a random improper coloring of the vertices with colors
from the allowed set, and then modifies this coloring by looking at the
vertices that forms a clash, that is have neighbours with the same colors. For
these vertices, every new coloring are evaluated in how many clashes they would
produce. Then vertex recoloring that has the lowest new number of clashes, even
if positive, is applied.

This can however introduce cycles, certain recolorings leading into each other
ad infinitum. To prevent this, the Tabu list is used. A Tabu list is a list 
of vertex-color pairs $(v,u)$. Whenever a new vertex recoloring is made, it is
added to the Tabu list. A new vertex recoloring is only considered if it's not a
part of the Tabu list. This tabu list have a set size, so that the first element
is removed when a new vertex-color pair is added. If there are more than one coloring 
that give the lowest increase in clashes, so is that broken randomly, also to avoid cycles.

Tabu moves are however allowed in some scenarios, specifically if applying that
coloring would create a better recoloring than the previous best, which is
commonly referred to as the aspiration. Whenever a new coloring is applied, the
current number of clashes are calculated, and the aspirations is set to that
number minus one.  A new coloring that is on the tabu list is then allowed
given that applying that coloring would result in a new number of clashes below
the current aspiration.

This process of picking a new vertex recoloring is repeated until a coloring
with zero clashes is produced, or a predetermined number of moves have been
made. As the algorithm only get's a valid coloring for a specific $k$, we also
must have a way to utilise this algorithm to get the lowest possible coloring.
Here a similar method described in \cite{Constructive} is used, that is, first a coloring
is made with RLF, and assign then number of colors in that coloring to $k$ .
The then we try to color it with TabuCol with that amount of colors minus one,
and subtract $k$ by one overtime we succeed, until we fail, at which point we
say that the current $k$ is the lowest we can go.

TabuCol has some subtle variations, and this particular definition is based on
the presentation in \cite{Constructive}. Alternatives are for example whether or not all
possible recolorings are consider, if only the first coloring with a lower clash
count is considered, and if one allows vertices that don't have any clashes to
be recolored.

\todo[noline]{Define clashes}
\begin{algorithm}[H]
    \caption{Clashes}
    \algsetup{linenodelimiter=}
    \begin{algorithmic}[1]
        \REQUIRE Improper coloring $\sigma$
        \REQUIRE Graph $G = (V,E)$
        \RETURN{$|\{(u,v) :(u,v) \in E, \sigma(u) = \sigma(v)\}|$}
    \end{algorithmic}
\end{algorithm}

\begin{algorithm}[H]
    \caption{TabuCol}
    \algsetup{linenodelimiter=}
    \begin{algorithmic}[1]
        \REQUIRE Graph $G = (V,E)$
        \REQUIRE Integer $k > 0$
        \REQUIRE Integer $MaxIt > 0$
        \REQUIRE Integer $MaxTabu > 0$
      
        \STATE $\sigma \leftarrow \text{random improper coloring with $k$ colors}$
        \STATE $CurIt \leftarrow 0$
        \STATE $CurrentClash = \textbf{Clashes}(\sigma,G)$
        \STATE $Asp \leftarrow CurrentClash-1$
        \STATE $Tabu \leftarrow \text{Empty tabu list}$
        \WHILE{$CurrentClash > 0 $ and $CurIt < MaxIt$}
            \STATE $Reps \leftarrow \emptyset$
            \FORALL{$v \in V$}
                \FORALL{$c \in  \sigma$}
                    \IF{$(v,c) \notin Tabu$ or $\textbf{Clashes}(\sigma,G) \text{ with $(v,c)$
                    applied}\leq Asp$}
                        \STATE $Reps \leftarrow Reps \cup \{(v,c)\}$
                    \ENDIF
                \ENDFOR
            \ENDFOR
            \STATE $(v',u') \leftarrow \text{where $(v',u') \in Reps$ and
            $\textbf{Clashes}(\sigma,G)$ with $(v',u')$ applied is minimal, choose randomly among ties }$
            \STATE Update $\sigma$ so that $\sigma(v') = u'$
            \STATE $CurrentClash \leftarrow \textbf{Clashes}(\sigma,G)$
            \IF{$CurrentClash \leq Asp $}
                \STATE $Asp \leftarrow CurrentClash -1$
            \ENDIF
            \STATE Update $Tabu$ to contain $(v',u')$, and remove the oldest element if $|Tabu| > MaxTabu$
            \STATE $CurIt \leftarrow CurIt + 1$
            \IF{$CurrentClash = 0$}
                \BREAK
            \ENDIF
        \ENDWHILE
    \end{algorithmic}
\end{algorithm}


\section{Strategies}
\label{sec:Strategies}
In the case where the whole graph doesn't contain any prime modules, coloring
with \autoref{alg:generic} is in linear time (REFERENS). It is when coloring prime modules
that we have to apply the heuristics, and these different methods are
described in this part.

\subsection{Whole graph}

In this baseline test, the whole graph is colored using the heuristic, not
utilizing the modular decomposition at all. This strategy is used as a baseline
to compare whether or not the other strategies improves the performance for the
graph.

\subsection{Whole prime coloring}

With this strategy, the prime vertex with the strong module $X$ of $G$, the
whole induced subgraph $G[X]$ is colored with the heuristic.

This method is the simplest combining strategy, but worth noting is that is
equivalent to coloring the whole graph using the heuristic in the case where the
root node is prime, and therefore only offers a possible improvement to existing
heuristics when the root in the modular decomposition isn't prime.

\subsection{Quotient recoloring}

This coloring, unlike the previous two, also attempts to colorize the prime
modules internally. Here, every prime module is colored just like in 'Whole
prime coloring', but only if it contains under a predetermined number of
vertices. Otherwise, all of it's children are colored first, then the quotient
graph for the children of the prime modules is constructed. This quotient graph
is then colored using RLF, and then every child module with the same
color in the quotient graph can now be recolored to use the same set of colors.

The quotient graph is graph describing whether or not vertex partitions of a
graph are adjecent or not, instead of individual vertices. Two vertex
partitions are adjecent in the quotient graph if there exists an edge between
any pair of vertices, where one part is in the first partition, and the other
part is in the other partition. A quotient graph where the partition is all of
the original vertices individually, is therefore homomorphic to the original
graph.

\begin{definition}[Quotient graph]
    The quotient graph $Q$ for a graph $G = (V,E)$ over partition 
    $P = \{P_1 \cdots P_i\}$  of the vertices $V$, is a graph 
    $Q = \{P, \{(P_i,P_j) : j\neq i, \exists v \in P_i,\exists u \in P_j( (u,v)
    \in E)   \}   $
\end{definition}

\todo[noline]{Add pseudocode}
\begin{algorithm}[H]
  \caption{Quotient color prime module}
  \algsetup{linenodelimiter=}
  \begin{algorithmic}[1]
    \REQUIRE Graph $G$
    \REQUIRE Modular decomposition $MD$
    \REQUIRE Module to color $M$
    \REQUIRE Threshold for heuristic $T$

    \IF{$|M| \leq T$}
        \RETURN{Color $M$ with an arbitrary color}
    \ENDIF
    \FORALL {child module $m$ of $M$ in $MD$}
        \STATE Apply \textbf{Quotient color prime module} on $m$
    \ENDFOR
    \STATE $Q \leftarrow $ quotient graph of $G[M]$ where the child modules of $M$ in $MD$ is the partition
    \STATE Color $Q$ with another heuristic

    \FORALL{$c \in \chi(Q)$}
        \STATE $Q_c \leftarrow \{ v : v \in V(Q), \sigma(v) = c\}$
        \STATE $m' \leftarrow$ the module in $Q_c$ with the most colors
        \STATE $S \leftarrow$ coloring used by $m'$, $\sigma(G[m'])$
        \FORALL {$m \in Q_c \setminus \{m'\} $}
            \STATE randomly choose an injective map $\phi:\sigma(m)\to S$
            \FORALL{$x \in m$}
                \STATE $\sigma(x) \leftarrow \phi(\sigma(x))$
            \ENDFOR
        \ENDFOR
    \ENDFOR

  \end{algorithmic}
\end{algorithm}

The threshold used for the test is 20 vertices.

\section{Data generation}
\label{sec:Data}

The test sets used are part graphs from the DIMACS benchmark set \cite{DIMACS}, 
and part graphs generated that aren't prime.

All of the graphs from the DIMACS benchmark sets are difficult to color graphs,
and also have modular decomposition where the root vertex is prime. As they are
commonly used benchmarks, they also provide known best current colorings for the
different graphs. However, the modular decomposition might provide a more
efficient way to color graphs where only some of the child nodes are prime. 

The algorithm for generating these graphs is described in \autoref{alg:RDCG}.
First an ordinary binary tree is randomly generated with a specified amount of
leafs , and then every label in this tree is given a label "series" or
"parallel".  This then describes a cograph, where the leafs are the K1 bases,
and a label of "series" means that the children are joined by disjoint union
and a label of "parallel" means that the children are joined by graph join.
From this so can the corresponding graph be constructed. From this graph, we
also randomly add edges in predetermined number of modules and with a
predetermined size. This then forms a graph with a modular decomposition which
doesn't contain a prime module as the root, but still contains prime modules.

\subsection{Random graph generation}

\begin{algorithm}[H]
    \caption{Construct cograph}
    \algsetup{linenodelimiter=}
   \begin{algorithmic}[1]
        \REQUIRE Vertex labeled tree $T$
        \IF{$|V(T)| = 1$}
            \RETURN{$(V(T),\emptyset)$}
        \ENDIF
        \STATE $G \leftarrow \text{empty graph $(\emptyset,\emptyset)$}$
        \FORALL{$v \in N_T(root(T))$}
            \IF{Label of $root(T)$ is \textit{series}}
                \STATE $G \leftarrow \bigcup G, \text{\textbf{Construct cograph} applied
                on the subtree of $T$ with $v$ as root} $
            \ELSIF{Label of $root(T)$ is \textit{parallel}}
                \STATE $G \leftarrow \nabla G, \text{\textbf{Construct cograph} applied
                on the subtree of $T$ with $v$ as root}$
            \ENDIF
        \ENDFOR
    \end{algorithmic}
\end{algorithm}

\begin{algorithm}[H]
    \caption{Random disturbed cograph generation}
  \algsetup{linenodelimiter=}
  \label{alg:RDCG}
  \begin{algorithmic}[1]
      \REQUIRE Percent $p$ for serial, $0 \neq p \neq 1$.
      \REQUIRE Percent $pe$ for new  edge, $0 \neq pe \neq 1$.
      \REQUIRE Total amount of leafs $l$, $0 < l$.
      \REQUIRE Prime modules size $ms$
      \REQUIRE Prime modules count $mc$

      \STATE $bg \leftarrow \text{Random binary graph with leaf count equal to $l$}$
      \FORALL{$v \in bg$}
        \STATE Randomly assign a label  \textit{series}, or \textit{parallel} to $v$, so
        that the probability for \textit{series} is $p$.
      \ENDFOR

      \STATE $CG \leftarrow \text{\textbf{Construct graph} applied on $bg$}$

      \STATE $pm \leftarrow \text{All strong modules $M$ of $CG$ such that $|M| \geq ms$, ordered in increasing order by size}$
      \FOR{$i \cdots mc$}
        \STATE $cm \leftarrow \text{The module at index $i$ in $pm$}$
        \FORALL{vertex pairs $(u,v)$ where $u,v \in V(CG[cm])$}
            \STATE Modify $CG$ by adding edge $(u,v)$ with a chance of $pe$
        \ENDFOR
      \ENDFOR
      \RETURN{$CG$}
  \end{algorithmic}
\end{algorithm}

The cographs are disturbed in such a way that the expected amount of prime
modules and size of these modules can be tuned beforehand. By only adding edges
within a module, we can guarantee that only vertices in that module can be part
of a new prime module. By specifying how many modules we want and their size, we
can also construct modular decomposition with various percentage prime modules.


\section{Evaluation}

There are a number of parameters that affect the generated graph, mainly the
number of vertices in the generate graph, the percent of "serial" when generatin
a random cograph, and the percent for edges within a module, and finally the
amount and size of the prime modules.

The graphs generated where of four different sizes, 250, 500, 750 and 1000. 
Within these categories so were also a few different configurations of graphs
tested. Whether or not many smaller or fewer large prime modules is better 
when applying locally is tested by having either 10 or 5 prime modules, with a
size so that amount*size is half of the size of the graph. The edge probability
within modules is a  constant
50\%, so that the different modules have roughly the same egde
density. The final split is with serial percent. A higher serial percent yields
a graph with more edges, and lower a graph with fewer  edges. The serial percent
tested where 35\% and 70\%, resulting in a total of 16 different combinations,
aswell as the DIMACS test set. The naming convention of the graphs are such that 
the first number represents the size, the second the serial percent,
and the third the amount of prime modules. All categories contain 15 graphs, 
so that they can be colored in paralell efficiently.

The average time it took to color the graph is also displayed. Note however that
system load may vary between different runs, and that because of the long
runtime for some test sets so are the tests not repeated to ensure statistical
significance. The time should therefore be seen as accurate only when displaying
large relative differences in time taken.

Finally so does TabuCol contain some parameters that can be tuned. The amount of
iterations allowed was for these test were set to 10000.

\section{Results and evaluation}
\label{sec:Result}
\todo[noline]{More everything, very unfinished}
The tables result the averaged performance for coloring the graphs in the
respective test sets. All graphs are colored with all combinations of heuristic
and combination strategy. A final table displaying the total average between all
graphs is also included.

%\includegraphics[width=\textwidth]{Results.png}
%\includegraphics[width=\textwidth]{TestGGSave.png}
\subsection{Results}
\input{Tables/DIMACSResult.tex}
\input{Tables/DisturbedCoGraph_1000_35_10Result.tex}
\input{Tables/DisturbedCoGraph_1000_35_5Result.tex}
\input{Tables/DisturbedCoGraph_1000_70_10Result.tex}
\input{Tables/DisturbedCoGraph_1000_70_5Result.tex}
\input{Tables/DisturbedCoGraph_250_35_10Result.tex}
\input{Tables/DisturbedCoGraph_250_35_5Result.tex}
\input{Tables/DisturbedCoGraph_250_70_10Result.tex}
\input{Tables/DisturbedCoGraph_250_70_5Result.tex}
\input{Tables/DisturbedCoGraph_500_35_10Result.tex}
\input{Tables/DisturbedCoGraph_500_35_5Result.tex}
\input{Tables/DisturbedCoGraph_500_70_10Result.tex}
\input{Tables/DisturbedCoGraph_500_70_5Result.tex}
\input{Tables/DisturbedCoGraph_750_35_10Result.tex}
\input{Tables/DisturbedCoGraph_750_35_5Result.tex}
\input{Tables/DisturbedCoGraph_750_70_10Result.tex}
\input{Tables/DisturbedCoGraph_750_70_5Result.tex}

\subsection{Discussion}

Something that we can see immediately from the tables, is that applying a
heuristic locally on prime modules doesn't significantly affect the resulting
color for RLF,Greedy and DSatur. It does however seem to give a minor
improvement for tabucol. The increase in performance is minor however, but does
seem to scale with the complexity of the graph.

When it comes to performance, we can see that for RLF, DSatur and Greedy, that
the runtime when applying locally on prime modules decreases with the size of
the graph.  As the non-prime parts of the graph can be colored in linear time by
the \autoref{alg:generic}, the prime modules left has a size smaller than the
total graph.  Algorithms whose runtime scales nonlinearly with the size of the
graph might can get a speed increase when applying locally only on the prime
modules, as demonstrated here.  This was however not the case for TabuCol, which
might be because of the high constant time setup for the algorithm, and because
the number of applications of the algorithm doesn't neccesarilly have clear
correlation with size of the graph.

Another advantage with the modular decomposition is that it could allow for more
easily implemented paralellisad coloring algorithms. As the modular creates a
partition of the graph vertices, so can the different parts of the modular
decomposition tree be colored in paralell, which might for example compensate
for the increased runtime when for example using TabuCol.

Using the modular decomposition for a speed increase would however require a
fast modular decomposition implementation.  While modular decomposition can be
done in linear time (REFERENS), implementing this has proven to be difficult in
practice (REFERENS). The baseline time for creating the modular decomposition
would most likely only be offset for larger graphs, which is also where a
linear time algorithm is required to save time.

\printbibliography

\end{document}
