
%&LaTeX
\documentclass[11pt]{beamer}
\usepackage[english]{babel} % Anpassa efter svenska. Ger svensk logga.
%\usepackage[swedish]{babel} % Anpassa efter svenska. Ger svensk logga.
\usepackage[latin1]{inputenc} % Anpassa efter linux
\hypersetup{pdfpagemode=FullScreen} % Adobe Reader ska ˆppna fullsk‰rm

\usetheme{SU} % Anpassa efter SU:s f‰rger och logga
\definecolor{aboveandunder}{named}{white} % Ger vit inramning
\usepackage{times} % Ger Times/Helvetica

\usepackage{amsmath} 
\usepackage{amssymb}
\usepackage{listings}
\usepackage[noend]{algorithmic}
\usepackage{algorithm}
%\usepackage{algorithm2e}

\author{Emanuel Bergren}
\title{Graph coloring using the modular decomposition}


\newcommand{\CenterPic}[2][<1->]
{
    \begin{figure}
        \centering
        \includegraphics#1[width=\textwidth,height=0.9\textheight,keepaspectratio]{#2}
    \end{figure}
}


\begin{document}
\begin{frame}
\titlepage
\end{frame}


\section{Graphs and graph coloring}

\begin{frame}
    \frametitle{Graph's}
    \CenterPic{GraphExample.pdf}
    \begin{itemize}
        \item set of vertices
        \item set of edges between vertices
    \end{itemize}
\end{frame}


\begin{frame}
    \frametitle{Graph Coloring}
    \begin{columns}[c]
        \begin{column}{0.5\textwidth}
            \CenterPic{./ColoringExample1.pdf}
        \end{column}
        \begin{column}{0.5\textwidth}
            \CenterPic{./ColoringExample2.pdf}
        \end{column}

    \end{columns}

\end{frame}

\section{Applications of graph coloring}

\begin{frame}

    \frametitle{Applications of graph coloring}
    \only<1-1>{\CenterPic{./RegisterAllocation.png}}
    \only<2-2>{\CenterPic{./Scheduling.png}}
    
\end{frame}


\begin{frame}
    
    \frametitle{Problems with graph coloring}

    \begin{itemize}
        \item Obtaining a coloring with the fewest colors is NP-Hard
        \item Worst case not polynomial time
        \item Optimal solutions to prohibitive to calculate in many applications, especially real-time/interactive.
    \end{itemize}
    
\end{frame}

\begin{frame}
    \frametitle{Heuristics}
    \begin{definition}[Graph coloring heuristic]
        A graph coloring heuristic is an algorithm that gives a graph a correct coloring, that is not necessarily optimal.
    \end{definition}
    \begin{itemize}
        \item Some heuristics are easy, assign every vertex a unique color, greedy.
        \item In general, trade off between better colorings and faster execution time
    \end{itemize}

\end{frame}

\begin{frame}
    
    The goal of this work is to develop a heuristic utilising the modular
    decomposition of a graph, that is used together with already existing heuristics.
    
\end{frame}

\section{Modular decomposition}

\subsection{Cographs}
\begin{frame}
    \begin{definition}[Cograph]
        A cograph if is a cograph if it is the graph join or disjoint union of 2
        cographs, or a graph consisting of a single vertex.
    \end{definition}
    \begin{definition}[Disjoint union]
        A cograph if is a cograph if it is the graph join or disjoint union of 2
        cographs, or a graph consisting of a single vertex.
    \end{definition}
    \begin{definition}[Graph join]
        A cograph if is a cograph if it is the graph join or disjoint union of 2
        cographs, or a graph consisting of a single vertex.
    \end{definition}

\end{frame}

\begin{frame}
    \frametitle{Graph join and disjoint union}
    
    \only<1-1>{\CenterPic{./GraphUnion_Pre.pdf}}
    \only<2-2>{\CenterPic{./GraphUnion_After.pdf}}
    \only<3-3>{\CenterPic{./GraphUnion_Pre.pdf}}
    \only<4-4>{\CenterPic{./GraphJoin_After.pdf}}
\end{frame}

\begin{frame}
    \frametitle{Example of cograph}
\end{frame}

\begin{frame}
    \frametitle{Properties of cographs}
    
    \begin{itemize}
        \item Most important here, optimal colorings for cographs can be found in linear time
        \item Not all graphs are cographs
    \end{itemize}
\end{frame}

\begin{frame}
    \frametitle{Modules}
    \begin{definition}[Module]
        A module is a subset of the vertices of a graph, that share the same
        neighbours excluding vertices within the module.
    \end{definition}
\end{frame}


\begin{frame}
    \frametitle{Example of modules}
    \only<1-1>
    {\CenterPic{./ModuleExample1.pdf}}
    \only<2-2>
    {\CenterPic{./ModuleExample2.pdf}}
    \only<3-3>
    {\CenterPic{./ModuleExample3.pdf}}
    %\only<4-4>
    %{\CenterPic{./ModuleExample4.pdf}}
\end{frame}


\begin{frame}
    \frametitle{Strong modules}
    \begin{definition}[Strong module]
        A module M is strong, if for every other module M', M is a subset of M' or M' is a subset of M, M and M' are disjoint.
    \end{definition}
\end{frame}

\begin{frame}
    \frametitle{Weak modules}
    Module $\{2,3\}$  and $\{2,4\}$ are not strong, as $\{2,3\} \nsubseteq \{2,4\}$ and $\{2,4\} \nsubseteq \{2,3\}$ while
    $\{2,3\} \cap \{2,4\} \neq \{\}$.
    \begin{columns}[c]
        \begin{column}{0.5\textwidth}
            \CenterPic{./WeakModule2.pdf}
        \end{column}
        \begin{column}{0.5\textwidth}
            \CenterPic{./WeakModule1.pdf}
        \end{column}
    \end{columns}
\end{frame}

\begin{frame}
    \frametitle{Strong modules}
    \CenterPic{./StrongModule1.pdf}
\end{frame}


\subsection{Modular decomposition}
\begin{frame}
    \frametitle{The modular decomposition}


    \only<1-1>
    {
    The central concept of this work, but a bit technical definition
    \begin{definition}<1-1>[Modular decomposition tree]
        The modular decomposition tree $(\widetilde{T},\widetilde{t})$ of a graph $G = (V,E)$,is a rooted
        vertex labeled tree, such that every vertex is associated with a strong
        module $X$ in $G$.  A vertex in the tree with associated strong module $M$
        is a child to the vertex with associated strong module $M'$ if and only if,
        $M \subset M'$ and there is no other strong module $M''$ so that $M \subset
        M'' \subset M'$. Every vertex with associated module $X$ also has a label
        distinguishing 3 cases:
        \begin{enumerate}
            \item Parallel: $G[X]$ is disconnected.
            \item Series: $\overline{G[X]}$ is disconnected.
            \item Prime: $G[X]$ and $\overline{G[X]}$ is connected.
        \end{enumerate}
        The root of the tree must also have $V$ as the associated strong module, and all 
        strong modules also have to be a part of the tree.
    \end{definition}
    }
    
    \only<2->
    {
    \begin{itemize}
        \item The modular decomposition of a graph, is a tree describing how the graph can be constructed from it's strong modules*
        \item Every vertex in the tree has a label, and is associated with a strong module of the graph. The label describes how
            the graph induced by the module can be constructed from the vertex's children.
        \item There are 3 labels: 
            \begin{itemize}
                \item Parallel: The parent can be constructed through disjoint union on its children
                \item Series: The parent can be constructed through graph join on its children
                \item Prime: The parent is constructed through some other arbitrary way
            \end{itemize}
    \end{itemize}
    }
\end{frame}

\begin{frame}
    \frametitle{Properties of the modular decomposition}
    \begin{itemize}
        \item The modular decomposition tree for a graph is unique
        \item Modular decomposition trees with only prime and series vertices are cographs
        \item The vertices of siblings (vertices that share the same parent) are disjoint
        \item All graphs have a modular decomposition
    \end{itemize}
\end{frame}

\begin{frame}
    \frametitle{Example modular decomposition}
    
    \only<1-1>{\CenterPic{./Graph1.pdf}}
   
    \only<2->{
    \begin{columns}[c]
        \begin{column}{0.5\textwidth}
            \CenterPic{./Graph1.pdf}
        \end{column}
        \begin{column}{0.5\textwidth}
            \only<2-2>{\CenterPic{./Graph2.pdf}}
            \only<3-3>{\CenterPic{./Graph3.pdf}}
            \only<4-4>{\CenterPic{./Graph4.pdf}}
            \only<5-5>{\CenterPic{./Graph5.pdf}}
        \end{column}
    \end{columns}
}
\end{frame}
\section{Coloring algorithm}


\begin{frame}
    \frametitle{Base coloring algorithm}
    Recursively colors the graph using the modular decomposition, case by label. Applied on the root node of the graph,
    with an initial coloring giving each vertex a unique color
    \begin{algorithm}[H]
      \algsetup{linenodelimiter=}
      \begin{algorithmic}[1]
        \REQUIRE Vertex $v$ in graph $G$ and MD tree
           \STATE Recursively color all children of $v$ with this algorithm
           \IF {Module of $v$ has only a single element}
               \STATE Already optimally colored
               \RETURN
           \ENDIF
           \IF {label of $v$ is parallel} 
               \STATE Recolor children to use the same colors
           \ELSIF{label of $v$ is series} 
              \STATE Child modules cannot share colors, nothing to be done
           \ENDIF
      \end{algorithmic}
    \end{algorithm}
\end{frame}

\begin{frame}
    \frametitle{Coloring example: Base}
    \only<1-1>{\CenterPic{./ModuleExample1.pdf}}
    \only<2-2>{\CenterPic{./BaseColoring1.pdf}}
    \only<3-3>{\CenterPic{./BaseColoring2.pdf}}
    \only<4-4>{\CenterPic{./BaseColoring3.pdf}}
\end{frame}

\begin{frame}
    \frametitle{Coloring with prime modules}
    \begin{algorithm}[H]
      \algsetup{linenodelimiter=}
      \begin{algorithmic}[1]
        \REQUIRE Vertex $v$ in graph $G$ and MD tree
           \STATE Recursively color all children of $v$ with this algorithm
           \IF {Module of $v$ has only a single element}
               \STATE Already optimally colored
               \RETURN
           \ENDIF
           \IF {label of $v$ is parallel} 
               \STATE Recolor children to use the same colors
           \ELSIF{label of $v$ is series} 
              \STATE Child modules cannot share colors, nothing to be done
           \ELSIF{label of $v$ is prime} 
              \STATE ?
           \ENDIF
      \end{algorithmic}
    \end{algorithm}
\end{frame}

\begin{frame}
    \frametitle{Strategies}
    The strategies are the different way to fill in the blanks

    Every strategy is paired with another regular heuristic.
    \begin{itemize}
        \item WholeGraph: Apply the heuristic on the whole graph, not really a strategy but used in comparisons in the result section
        \item WholePrime: Apply heuristic on the subgraph induced by prime modules
        \item Quotient: Also try to color the prime modules with the modular decomposition using quotient graphs
    \end{itemize}
\end{frame}

\begin{frame}
    \frametitle{Coloring example: WholePrime}
    \only<1-1>{\CenterPic{./Graph5.pdf}}
    \only<2-2>{\CenterPic{./MDColoring1.pdf}}
    \only<3-4>
    {
        \begin{columns}[c]
            \begin{column}{0.5\textwidth}
                \CenterPic{./MDColoring6.pdf}
            \end{column}
            \begin{column}{0.5\textwidth}
                \only<3-3>{\CenterPic{./MDColoring2.pdf}}
                \only<4-4>{\CenterPic{./MDColoring3.pdf}}
            \end{column}
        \end{columns}
    }
    \only<5-5>
    {
        \CenterPic{./MDColoring4.pdf}
    }
    \only<6-6>
    {
        \CenterPic{./MDColoring5.pdf}
    }
\end{frame}

\begin{frame}
    \frametitle{Quotient coloring}
    WholePrime coloring does not utilise the modular decomposition to color prime modules, 
    Quotient coloring uses it.
    \begin{itemize}
        \item A quotient graph is a graph whose vertices represent a set of vertices from another graph, and edges represent that 2 set of vertices
            has some edge between them
        \item Idea: Color child modules of prime modules, and use a quotient graph to see if modules can share colors
        \item Cut-off point where prime modules are colored with regular heuristic, to get reasonable execution times
    \end{itemize}
\end{frame}

\begin{frame}
    \frametitle{Coloring example: Quotient}
    \only<1-1>{\CenterPic{./QuotientColoring1.pdf}}
    \only<2-2>{\CenterPic{./QuotientColoring2.pdf}}
    \only<3-3>{\CenterPic{./QuotientColoring3.pdf}}
    \only<4-4>{\CenterPic{./QuotientColoring4.pdf}}
    \only<5-7>
    {
        \begin{columns}[c]
            \begin{column}{0.5\textwidth}
                \only<5-6>{\CenterPic{./QuotientColoring9.pdf}}
                \only<7-7>{\CenterPic{./QuotientColoring7.pdf}}
            \end{column}
            \begin{column}{0.5\textwidth}
                \only<5-5>{\CenterPic{./QuotientColoring5.pdf}}
                \only<6-6>{\CenterPic{./QuotientColoring6.pdf}}
                \only<7-7>{\CenterPic{./QuotientColoring6.pdf}}
            \end{column}
        \end{columns}
    }
    %\only<5-5>{\CenterPic{./QuotientColoring5.pdf}}
    %\only<6-6>{\CenterPic{./QuotientColoring6.pdf}}
    %\only<7-7>{\CenterPic{./QuotientColoring7.pdf}}
    \only<8-8>{\CenterPic{./QuotientColoring8.pdf}}
    %\only<2-2>{\CenterPic{./MDColoring1.pdf}}
    %\only<3-4>
    %{
    %    \begin{columns}[c]
    %        \begin{column}{0.5\textwidth}
    %            \CenterPic{./MDColoring1.pdf}
    %        \end{column}
    %        \begin{column}{0.5\textwidth}
    %            \only<3-3>{\CenterPic{./MDColoring2.pdf}}
    %            \only<4-4>{\CenterPic{./MDColoring3.pdf}}
    %        \end{column}
    %    \end{columns}
    %}
    %\only<5-5>
    %{
    %    \CenterPic{./MDColoring4.pdf}
    %}
    %\only<6-6>
    %{
    %    \CenterPic{./MDColoring5.pdf}
    %}
\end{frame}

\begin{frame}
    \frametitle{Heuristics}
    Ordered by fastest to slowest
    \begin{itemize}
        \item Greedy
        \item DSatur
        \item RLF
        \item TabuCol
    \end{itemize}
\end{frame}

\begin{frame}
    \frametitle{Test data}
    \begin{itemize}
        \item Large graphs with prime root: DIMACS
        \item Randomly generated graphs with non-prime root
            \begin{itemize}
                \item Graph size
                \item Amount of vertices in prime modules
                \item Edge density 
            \end{itemize}
    \end{itemize}
\end{frame}

\section{Results}

\newcommand{\DisCol}[2]
{
    \begin{columns}[c]
        \begin{column}{0.5\textwidth}
            \CenterPic{#1}
        \end{column}
        \begin{column}{0.5\textwidth}
            \CenterPic{#2}
        \end{column}
    \end{columns}
}

\begin{frame}
    \frametitle{Results}
    \only<1-1>{\DisCol{../Paper/Tables/250.png}{../Paper/Tables/250Time.png}}
    \only<2-2>{\DisCol{../Paper/Tables/500.png}{../Paper/Tables/500Time.png}}
    \only<3-3>{\DisCol{../Paper/Tables/750.png}{../Paper/Tables/750Time.png}}
    \only<4-4>{\DisCol{../Paper/Tables/1000.png}{../Paper/Tables/1000Time.png}}
    \only<5-5>{\DisCol{../Paper/Tables/DIMACS.png}{../Paper/Tables/DIMACSTime.png}}
\end{frame}


\begin{frame}
    \frametitle{Interpretation of the results}
    \begin{itemize}
        \item Minimal difference for most algorithms, slight improvements for TabuCol
        \item TabuCol is already slow, might be worth it
        \item Could potentially be used to speed up calculations, especially with multi-threading
        \item Quotient coloring never improved performance beyond the underlying algorithm
    \end{itemize}
\end{frame}

\end{document}
